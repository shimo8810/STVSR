\documentclass[twocolumn,a4j,uplatex]{jsarticle}
\usepackage{esys-thesis} %筑波のスタイル
\usepackage[dvipdfmx]{graphicx} %garphix
\usepackage{amsmath,amssymb} %ams系
\usepackage[T1]{fontenc} %T1 エンコーディング
\usepackage{newtxtext, newtxmath} %Time系フォント
\usepackage{ascmac} % 特殊記号
\usepackage{cite} % 引用
\usepackage{url} % URL関連
\usepackage[dvipdfmx]{hyperref} %ハイパーリンク関連
\newcommand{\argmax}{\mathop{\rm arg~max}\limits}
\renewcommand{\figurename}{Fig.}
\renewcommand{\tablename}{Table }
\makeatletter
\renewcommand{\thepage}{知能機能システム専攻セミナー -\arabic{page}}
\makeatother
%論文本稿
\pagestyle{myheadings}
% タイトル
\title{{\Large 多様な誤差関数を用いた深層学習による動画像の時空間超解像} \\
Spatio-Temporal Video Super Resolution by Deep Learning Using Various Loss Functions}
% auther
\author{{\Large 下平 勇斗} \\ Hayato SHIMODAIRA \\ (指導教員 延原 肇, 丸山 勉、北原 格)}

\begin{document}
\maketitle
\thispagestyle{headings}
\section{はじめに}
% 書く内容
% 将来的には4k8kが必要となる , 総務省が言ってるからな
% 超解像とかフレーム補間が必要になる
% 数年前の超解像研究とかフレーム補間研究では従来の手法が使われていた
% また近年深層学習を用いた学習が普及している
% そこでフレーム補間と超解像を行うネットワークを構築する
現在の日本におけるテレビ放送は地上デジタル放送で2K解像度での放送を行っているが,
総務省によると2020年の東京オリンピック・パラリンピックに向けて,
衛星放送やケーブルテレビ等を通して高解像度な4K,8K解像度の実用放送を開始する計画であり,
2018年には衛星放送での4K,8K解像度の実用放送を開始する予定である~\cite{soumu}.

しかし, 4K解像度の4K UHDTV(3840x2160)と比較して
4K以前の映像作品の解像度はHDTV(1920x1080)で25\%, SDTV(720×480)で約4.2\%程度となり,
過去の映像作品では将来のディスプレイの解像度に対応するために解像度を拡張する必要がある.
この様な解像度を拡張する技術を超解像と呼び, 画像処理の一般の問題として広く知られている.
% ToDo: 超解像技術は単に同画像の拡張だけでなく, 云々をここに加える

また, 旧来の映像作品は24fpsのフレームレートの映像作品が主流であるが,
ディスプレイ性能の向上や映像作品の配信サービスの普及などによりフレームレートの向上に需要がある.
フレーム数を増加させるためにはフレーム間の中間画像を生成する必要がある.
この様な問題をフレーム補間と呼ぶ.
% アニメーションの中割にも適用できるとか
フレーム補間や超解像などの画像処理の問題は古くから様々な手法で研究されてきたがその中でも,
深層学習を用いた研究が昨今盛んに行われている.本研究でもフレーム補間と超解像を深層学習を用いて行う.

深層学習を用いた超解像やフレーム補間モデルは提案されているがその双方を行うモデルは無い.
また先行研究で述べるように, 最小二乗誤差を誤差関数に用いることによる生成された画像がぼやけてしまう問題がある.
これらの問題に対して, 最小二乗誤差以外の誤差関数も用いた深層学習モデルを提案する.

\section{先行研究}
% 書く内容
% SRCNN
深層学習を用いた超解像の研究ではDongらのSRCNN\cite{Dong2015}がある.
この研究では2, 3層のCNN層を用いるだけで既存の深層学習を
使用しない超解像手法よりもPSNR, SSIMなどの複数の画像の評価手法で高精度であった.
またこの研究ではCNN層を単純に増加させるだけでは精度は向上しない事が述べられており,
複雑なネットワークの必要性を示唆している.

% SDRL
Heらのまた物体認識で高い識別精度を出したResNet\cite{He_2016_CVPR}で提案された
Skip Connectionの構造は超解像やフレーム補間のネットワークにも適応され,
超解像ではKimらのVDSR\cite{Kim_2016_CVPR}で適応されており,
高解像画像と低解像画像の差分のスパース画像を学習するような構造になっている.

% 対称性AEを用いている
% Synmetric-ED
Maoらは対称なSkip Connectionを用いたEncoder-Decoderモデルを提案した~\cite{DBLP:conf/nips/2016}.
この研究では単一画像の超解像の為のネットワークを提案し,
このネットワークではResNet\cite{He_2016_CVPR}の構造を取り入れResNetのSkip Conectionを
Encoder-Decoderの層で対称になるように接続している.

本研究でもSkip Connectionを用いたネットワーク構造を使用する.

% Beyond MSR
出力が画像となる様な深層学習のモデルでは誤差関数は平均二乗誤差(MSE)を一般に採用されるが,
がMathieuら\cite{1511.05440}やLedigらの\cite{Ledig2016arxiv}の論文で
平均二乗誤差を誤差関数とすることは高解像度の生成に寄与しない.
平均二乗誤差では入力画像(低解像画像)に対応する正解画像(高解像画像)が必ずしも1つではなく
複数存在するため, その複数の平均的な画像つまりぼやけた画像が生成されてしまう問題がある事が述べられている.
\section{提案手法}
\subsection{データ構造}
\begin{figure}[htbp]
    \centering
    \includegraphics[width=0.9\linewidth]{../../images/data.pdf}
    \caption{動画シーケンスデータの構造}
    \label{fig:dataset}
\end{figure}

学習に用いる動画シーケンスデータは図\ref{fig:dataset}の通り,入力画像群と出力画像の対である.
元々の動画データから$N_f + 1$フレーム分だけ取り出し,
取り出した画像群の中心の画像を正解画像, その他の画像群を入力画像群とする.
入力画像群は合計で$N_f$フレーム存在し, それぞれのフレームを$f_i~(i = i,  2, \cdots , N_f)$とする.
この$N_f$フレームの入力画像群から中心の正解画像を生成する.
\subsection{モデル}
\begin{figure}[htbp]
    \centering
    \includegraphics[width=0.9\linewidth]{../../images/model.pdf}
    \caption{学習モデルの構造}
    \label{fig:model}
\end{figure}
ネットワークのモデルは文献\cite{DBLP:conf/nips/2016}のモデルを基礎にしたモデルを用いている.
モデルは図\ref{fig:model}の通り, 中間層のサイズを変えないCNNとダウンサンプリングを含むCNNの対が複数含まれるEncoder層と
中間層のサイズを変えないCNNとアップサンプリングを含むCNNの対が複数含まれるDecoder層,
そしてそれらをつなぐSkip Connectionによって構成されている.
このSkip Connectionは層の出力と足し合わせるのではなく層をconcatしている.
活性化関数にはReLUを使用している.
\subsection{誤差関数}
\begin{figure}[htbp]
    \centering
    \includegraphics[width=0.9\linewidth]{../../images/vgg16.pdf}
    \caption{VGG16モデルを用いた誤差関数}
    \label{fig:vgg16loss}
\end{figure}

誤差関数には通常の平均二乗誤差$L_{MSE}$だけでなく
図\ref{fig:vgg16loss}のようなVGG16\cite{VGG16}の中間層を誤差とする誤差関数を用いた.
これは先に述べられたとおり, 平均二乗誤差ではぼやけた画像が生成されてしまう問題が存在した.
このためより画像の特徴に着目した誤差関数が必要であると考えた.
物体の分類を識別するネットワークのFeature Mapでは浅い層では勾配方向や色などの情報が, 深い層ではより抽象的な情報が
表現されている. これらの情報が生成された画像と正解画像で一致することによってより高精度の画像生成が行えると考え,
既に学習済みの識別モデルのそれぞれの中間層を比較しそれを誤差関数とした.
各Feature Map間の誤差は平均二乗誤差を使用した.
ここでは学習済みの識別モデルとしてVGG16\cite{VGG16}を使用した.

\section{実験}
提案手法をフレーム補間に適応できるかを実験した.
学習には動画分類タスク用のデータセットであるUCF101\cite{UCF101}を使用した.
実装にはPythonとChainer\cite{chainer_learningsys2015}を使用し, 計算にはGeForce GTX 1080 Tiを使用した.
最適化アルゴリズムはAdam\cite{kingma2014adam}を使用した.

実験では
\section{結論}

\bibliographystyle{junsrt}
\bibliography{biblio/web_page}
\end{document}

