\documentclass[twocolumn,a4j,uplatex]{jsarticle}
\usepackage{esys-thesis} %筑波のスタイル
\usepackage[dvipdfmx]{graphicx} %garphix
\usepackage{amsmath,amssymb} %ams系
\usepackage[T1]{fontenc} %T1 エンコーディング
\usepackage{newtxtext, newtxmath} %Time系フォント
\usepackage{ascmac} % 特殊記号
\usepackage{cite} % 引用
\usepackage{url} % URL関連
\usepackage[dvipdfmx]{hyperref} %ハイパーリンク関連
\newcommand{\argmax}{\mathop{\rm arg~max}\limits}
\makeatletter
\renewcommand{\thepage}{知能機能システム専攻セミナー -\arabic{page}}
\makeatother
%論文本稿
\pagestyle{myheadings}
% タイトル
\title{{\Large 多様な誤差関数を用いた深層学習による動画像の時空間超解像} \\
Spatio-Temporal Video Super Resolution by Deep Learning Using Various Loss Functions}
% auther
\author{{\Large 下平 勇斗} \\ Hayato SHIMODAIRA \\ (指導教員 延原 肇, 丸山 勉、北原 格)}

\begin{document}
\maketitle
\thispagestyle{headings}
\section{概要}
% 書く内容
% 将来的には4k8kが必要となる , 総務省が言ってるからな
% 超解像とかフレーム補間が必要になる
% 数年前の超解像研究とかフレーム補間研究では従来の手法が使われていた
% また近年深層学習を用いた学習が普及している
% そこでフレーム補間と超解像を行うネットワークを構築する
現在の日本におけるテレビ放送は地上デジタル放送で2K解像度での放送を行っているが,
総務省によると2020年の東京オリンピック・パラリンピックに向けて,
衛星放送やケーブルテレビ等を通して高解像度な4K,8K解像度の実用放送を開始する計画であり,
2018年には衛星放送での4K,8K解像度の実用放送を開始する予定である~\cite{soumu}.

しかし, 4K解像度の4K UHDTV(3840x2160)と比較して
4K以前の映像作品の解像度はHDTV(1920x1080)で25\%, SDTV(720×480)で約4.2\%程度となり,
過去の映像作品では将来のディスプレイの解像度に対応するために解像度を拡張する必要がある.
この様な解像度を拡張する技術を超解像と呼び, 画像処理の一般の問題として広く知られている.
% ToDo: 超解像技術は単に同画像の拡張だけでなく, 云々をここに加える

また, 旧来の映像作品は24fpsのフレームレートの映像作品が主流であるが,
ディスプレイ性能の向上や映像作品の配信サービスの普及などによりフレームレートの向上に需要がある.
フレーム数を増加させるためにはフレーム間の中間画像を生成する必要がある.
この様な問題をフレーム補間と呼ぶ.
% アニメーションの中割にも適用できるとか
フレーム補間や超解像などの画像処理の問題は古くから様々な手法で研究されてきたがその中でも,
深層学習を用いた研究が昨今盛んに行われている.本研究でもフレーム補間と超解像を深層学習を用いて行う.

本稿ではフレーム補間と超解像を複合し時空間超解像と呼び, 時空間超解像を行う深層学習を提案する.
\section{先行研究}
% 書く内容
% SRCNN
深層学習を用いた超解像の研究ではDongらのSRCNN\cite{Dong2015}がある.
この研究では2, 3層のCNN層を用いるだけで既存の深層学習を
使用しない超解像手法よりもPSNR, SSIMなどの複数の画像の評価手法で高精度であった.
またこの研究ではCNN層を単純に増加させるだけでは精度は向上しない事が述べられており,
複雑なネットワークの必要性を示唆している.

% SDRL
Heらのまた物体認識で高い識別精度を出したResNet\cite{He_2016_CVPR}で提案された
Skip Connectionの構造は超解像やフレーム補間のネットワークにも適応され,
超解像ではKimらのVDSR\cite{Kim_2016_CVPR}で適応されており,
高解像画像と低解像画像の差分のスパース画像を学習するような構造になっている.

% 対称性AEを用いている
% Synmetric-ED
Maoらは対称なSkip Connectionを用いたEncoder-Decoderモデルを提案した~\cite{DBLP:conf/nips/2016}.
この研究では単一画像の超解像の為のネットワークを提案し,
このネットワークではResNet\cite{He_2016_CVPR}の構造を取り入れResNetのSkip Conectionを
Encoder-Decoderの層で対称になるように接続している.

本研究でもSkip Connectionを用いたネットワーク構造を使用する.

% Beyond MSR
出力が画像となる様な深層学習のモデルでは誤差関数は平均二乗誤差(MSE)を一般に採用されるが,
\subsection*{DEEP MULTI-SCALE VIDEO PREDICTION BEYOND MEAN SQUARE ERROR}
\section{提案手法}
\section{実験}
\section{結論}

\bibliographystyle{junsrt}
\bibliography{biblio/web_page}
\end{document}

