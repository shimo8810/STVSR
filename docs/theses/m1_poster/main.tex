\documentclass[twocolumn,a4j,uplatex]{jsarticle}
\usepackage{esys-thesis} %筑波のスタイル
\usepackage[dvipdfmx]{graphicx} %garphix
\usepackage{amsmath,amssymb} %ams系
\usepackage[T1]{fontenc} %T1 エンコーディング
\usepackage{newtxtext, newtxmath} %Time系フォント
\usepackage{ascmac} % 特殊記号
\usepackage{cite} % 引用
\usepackage{url} % URL関連
\usepackage[dvipdfmx]{hyperref} %ハイパーリンク関連
\newcommand{\argmax}{\mathop{\rm arg~max}\limits}
\makeatletter
\renewcommand{\thepage}{知能機能システム専攻セミナー -\arabic{page}}
\makeatother
%論文本稿
\pagestyle{myheadings}
% タイトル
\title{{\Large 多様な誤差関数を用いた深層学習による動画像の時空間超解像} \\
Spatio-Temporal Video Super Resolution by Deep Learning Using Various Loss Functions}
% auther
\author{{\Large 下平 勇斗} \\ Hayato SHIMODAIRA \\ (指導教員 延原 肇)}

% 概要
\eabstract{時空間動画超解像手法としてSTVSR}

\begin{document}
\maketitle
\thispagestyle{headings}
\section{概要}
% 書く内容
% 将来的には4k8kが必要となる , 総務省が言ってるからな
% 超解像とかフレーム補間が必要になる
% 数年前の超解像研究とかフレーム補間研究では従来の手法が使われていた
% また近年深層学習を用いた学習が普及している
% そこでフレーム補間と超解像を行うネットワークを構築する
現在の日本におけるテレビ放送は地上デジタル放送で2K解像度での放送を行っているが,
総務省によると2020年の東京オリンピック・パラリンピックに向けて,
衛星放送やケーブルテレビ等を通して高解像度な4K,8K解像度の実用放送を開始する計画であり,
2018年には衛星放送での4K,8K解像度の実用放送を開始する予定である~\cite{soumu}.

しかし, 4K解像度の4K UHDTV(3840x2160)と比較して
4K以前の映像作品の解像度はHDTV(1920x1080)で25\%, SDTV(720×480)で約4.2\%程度となり,
過去の映像作品では将来のディスプレイの解像度に対応するために解像度を拡張する必要がある.
この様な解像度を拡張する技術を超解像と呼び, 画像処理の一般の問題として広く知られている.
% ToDo: 超解像技術は単に同画像の拡張だけでなく, 云々をここに加える

また, 旧来の映像作品は24fpsのフレームレートの映像作品が主流であるが,
ディスプレイ性能の向上や映像作品の配信サービスの普及などによりフレームレートの向上に需要がある.
フレーム数を増加させるためにはフレーム間の中間画像を生成する必要がある.
この様な問題をフレーム補間と呼ぶ.
% アニメーションの中割にも適用できるとか
フレーム補間や超解像などの画像処理の問題は古くから様々な手法で研究されてきたがその中でも,
深層学習を用いた研究が昨今盛んに行われている.本研究でもフレーム補間と超解像を深層学習を用いて行う.

本稿ではフレーム補間と超解像を複合し時空間超解像と呼び, 時空間超解像を行う深層学習を提案する.
\section{先行研究}
% 書く内容
% 対称性AEを用いている
Xiao-Jiao Naoらは対称なSkip Conectionを用いたEncoder-Decoderモデルを提案した~\cite{DBLP:conf/nips/2016}.
この研究では単一画像の超解像の為のネットワークを提案し,
このネットワークではResNetの構造を取り入れResNetのSkip ConectionをEncoder-Decoderの層で対称になるように接続している.

% Beyond MSR
超解像の深層学習を用いた研究ではSRCNNという
また画像を生成する深層学習を
\subsection*{DEEP MULTI-SCALE VIDEO PREDICTION BEYOND MEAN SQUARE ERROR}
\section{提案手法}
\section{実験}
\section{結論}

\bibliographystyle{jplain}
\bibliography{biblio/web_page}
\end{document}

